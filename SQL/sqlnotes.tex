\documentclass[12pt,reqno]{amsart}

\usepackage{amsthm,amsmath,amssymb}
\usepackage{enumerate}
\usepackage[margin=0.5in]{geometry}
\usepackage{minted}
%\usepackage{hyperref}

\renewcommand{\baselinestretch}{1.3}
\setcounter{tocdepth}{1}

\usepackage{chngcntr}
\counterwithin{figure}{subsection}

\begin{document}

\title{SQL Programming Notes}
\author{Andrew Maurer}
\date{\today}

\begin{abstract}
  This is a bit more limited in scope than the other documents. But I'd like to learn SQL. Resources include the Khan Academy series on SQL.
\end{abstract}

\maketitle

\begin{center}
  \parbox{4.7in}{
    \tableofcontents
    }
\end{center}

\section{Basic Ideas}
\label{sec:sql-idea}

Databases store information in tables. It's inefficient to store all data in a single table. Relational databases like SQL have multiple tables, and make it easy to access data in one table from another table.

\section{Commands}
\label{sec:commands}

\subsection{Creating Databases}
\label{sec:create}

\begin{figure}[H]
  \centering
  \begin{minted}{sql}
CREATE TABLE groceries (id INTEGER PRIMARY KEY, name TEXT, quantity INTEGER);

INSERT INTO groceries VALUES (1, "Bananas", 4);
INSERT INTO groceries VALUES (2, "Peanut Butter", 1);
INSERT INTO groceries VALUES (1, "Chocolate Bars", 2);

SELECT * FROM groceries;
  \end{minted}
  \caption{Creating a table}
  \label{fig:create_table}
\end{figure}

\section{Aggregate Functions}
\label{sec:aggregate}

\appendix

\end{document}

%%% Local Variables:
%%% LaTeX-command: "latex -shell-escape"
%%% End: