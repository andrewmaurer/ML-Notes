\section{Data and Inference}
\label{sec:data}

\subsection{Data}

Generally speaking, data may be in one of two forms: numerical or categorical. Numerical data consist of scalars $x$ or vectors $\vec{x} \in \RR^p$. Categorical take on a discrete set of values in some set $S$.

The following are examples of numerical data:
\begin{enumerate}
\item 
\end{enumerate}

While these are examples of categorical data:
\begin{enumerate}
\item 
\end{enumerate}

There are ways to apply numerical techniques to categorical data. One such method is called \emph{one-hot encoding}. In this case, the elements of the discrete set $S$ are enumerated $s_1,\ldots,s_r$, and an observation of the event $s_j$ is encoded by the unit vector $e_j \in \RR^r$. This is a hugely inefficient process, and it is often adventageous to reduce the dimension of this numerical embedding of categorical data. Such techniques are covered in Section [??].

\subsection{Supervised Models}

The data presented in a \emph{supervised learning problem} will consist of data vectors $\vec{x}$ The goal when modeling is to choose a smoothly parametrized set of functions $F$, whose input resembles your data and whose output corresponds with the associated observations.

%%% Local Variables:
%%% TeX-master: "notes"
%%% LaTeX-command: "latex -shell-escape"
%%% End: